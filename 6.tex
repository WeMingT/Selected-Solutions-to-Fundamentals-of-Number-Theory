\section{第六周作业}
\subsection*{第四章}

\begin{enumerate}
    \item[13] \begin{enumerate}
        \item 求 $3^{400}$ 的最后一位数字;
        \item 求 $\left(12371^{56}+34\right)^{28}$ 被 111 除以后所得的余数.
    \end{enumerate}

\begin{solution}
    \begin{enumerate}
        \item $3^{400} \equiv 3^{4 \times 100} \equiv (3^4)^{100} \equiv 1^{100} \equiv 1 \pmod{10}$
        
    故其最后一位数字为1。
    
    \item  注意到 $12371 \equiv 111^2 + 50$,因此
    \begin{equation*}
        12371 \equiv 50 \pmod{111}
    \end{equation*}
    故
    \begin{equation*}
        (12371^{56} + 34)^{28} \equiv (50^{56} + 34)^{28} \pmod{111}
    \end{equation*}
    我们有
    \begin{align*}
    50^2 &\equiv 58 \pmod{111}\\
    50^4 &\equiv 34 \pmod{111}\\
    50^8 &\equiv 46 \pmod{111}\\
    50^{16} &\equiv 7 \pmod{111}\\
    50^{32} &\equiv 49 \pmod{111}
    \end{align*}
    因此
    \begin{equation*}
        50^{56} \equiv 50^{32} \times 50^{16} \times 50^8 \equiv 16 \pmod{111}
    \end{equation*}
    于是
    \begin{equation*}
        (50^{56} + 34)^{28} \equiv 50^{28} \pmod{111}
    \end{equation*}
    而
    \begin{equation*}
        50^{28} \equiv 50^{16} \times 50^8 \times 50^4 \equiv 70 \pmod{111}
    \end{equation*}
    综上,其余数为70。
    \end{enumerate}
\end{solution}

    \item[14] \begin{enumerate}
        \item 求 $3^{400}$ 的最后两位数字;
        \item 求 $9^{9^9}$ 的最后两位数字.
    \end{enumerate}

\begin{solution}
    \begin{enumerate}
        \item $\phi(100) = \phi(2^2 \times 5^2) = \phi(2^2) \phi(5^2) = (4-2) \times (25-5) = 40$
        
    由Euler定理知
    \begin{equation*}
        3^{\phi(100)} = 3^{40} \equiv 1 \pmod{100}
    \end{equation*}
    于是
    \begin{equation*}
        3^{400} \equiv (3^{40})^{10} \equiv 1 \pmod{100}
    \end{equation*}
    故其末两位为01.
    
    \item 我们有
    \begin{equation*}
        9^9 \equiv 9^{2 \times 4+1} \equiv (9^2)^4 \times 9 \equiv (1)^4 \times 9 \equiv 9 \pmod{40}
    \end{equation*}
    由Euler定理知
    \begin{equation*}
        9^{\phi(100)} = 9^{40} \equiv 1 \pmod{100}
    \end{equation*}
    因此
    \begin{equation*}
        9^{9^9} \equiv 9^9 \pmod{100}
    \end{equation*}
    对 9, 100 进行辗转相除法,有
    \begin{equation*}
        9^{-1} \equiv 89 \pmod{100}
    \end{equation*}
    而
    \begin{equation*}
        9^9 \times 9 = 9^{10} \equiv 1 \pmod{100}
    \end{equation*}
    故
    \begin{equation*}
        9^9 \equiv 9^{-1} \equiv 89 \pmod{100}
    \end{equation*}
    综上,其最后两位数字为89
    \end{enumerate}
\end{solution}

    \item[21] 当 $a$ 为何值时 $x^3 \equiv a(\bmod 9)$ 有解.
    
\begin{solution}
    遍历$\mathbb{Z} / 9 \mathbb{Z},$当且仅当 $a \equiv 0, 1, 8 \pmod{9}$ 时该方程有解。
\end{solution}

    \item[28] 设 $m_1, m_2, \cdots, m_K$ 两两互素,则同余方程组 $a_i x \equiv b_i\left(\bmod m_i\right), 1 \leqslant i \leqslant$ $K$ 有解的充要条件是每一个同余方程 $a_i x \equiv b_i\left(\bmod m_i\right)$ 均可解.
    
    \begin{proof}
        必要性显然,下证充分性。
        
        假设 $a_i x \equiv b_i \pmod{m_i}, 1 \leq i \leq k$ 的可解,则
        \begin{equation*}
            (a_i, m_i) \mid b_i
        \end{equation*}
        令 $d_i = (a_i, m_i)$, $a_i' = \frac{a_i}{d_i}$, $b_i' = \frac{b_i}{d_i}$, $m_i' = \frac{m_i}{d_i}$。
        
        则有
        \begin{equation*}
            a_i x \equiv b_i \pmod{m_i} \Leftrightarrow a_i' x \equiv b_i' \pmod{m_i'}
        \end{equation*}
        由于 $(a_i', m_i') = 1$,故方程有唯一一解
        \begin{equation*}
            x \equiv x_{i,0} \pmod{m_i'}
        \end{equation*}
        因此,原同余方程组等价于
        \begin{equation*}
            \begin{cases}
                x \equiv x_{1,0} \pmod{m_1'} \\
                x \equiv x_{2,0} \pmod{m_2'} \\
                \cdots \\
                x \equiv x_{k,0} \pmod{m_k'}
                \end{cases}
        \end{equation*}
        对于 $i, j \in \{1, 2, \cdots, k\}; i \neq j$,取素数 $p \mid (m_i', m_j')$,则
        $p \mid m_i$ 且 $p \mid m_j$,而 $(m_i, m_j) = 1$。故 $p$ 不存在。
        
        因此
        \begin{equation*}
            (m_i', m_j') = 1, 1 \leq i \neq j \leq k
        \end{equation*}
        由中国剩余定理知同余方程组有解。
    \end{proof}

    \item[29] 设 $(a, b)=1, C>0$ ,证明一定存在整数 $x$ ,使 $(a+b x, C)=1.$

    \begin{proof}
        证明:对$C$进行素因子分解
        \begin{equation*}
            C = p_1^{e_1} p_2^{e_2} \cdots p_k^{e_k}
        \end{equation*}
        构造如下的同余方程组
        \begin{equation*}
            x \equiv d_i \pmod{p_i}, i=1,2,\cdots,k
        \end{equation*}
        其中,
        \begin{equation*}
            d_i = \begin{cases}
                0, & p_i \nmid b \\
                -a b^{-1} + 1 \pmod{p_i}, & p_i \mid b
                \end{cases}
        \end{equation*}
        由于$p_1, p_2, \cdots, p_k$两两互素,由中国剩余定理知其有唯一解$x_0$。

        由方程组的构造知
        \begin{equation*}
            a + b x_0 \not\equiv 0 \pmod{p_i}, i=1,2,\cdots,k
        \end{equation*}
        因此
        \begin{equation*}
            (a + b x_0, C) = 1
        \end{equation*}
    \end{proof}
\end{enumerate}

\subsection*{第五章}

\begin{enumerate}
    \item[1] 设整数 $\alpha \geqslant 1, p$ 是奇素数,若 $p^\alpha \nmid a$ ,求
    $$
    x^2 \equiv a\left(\bmod p^\alpha\right)
    $$
    的一切解.

\begin{solution}
    (1) 若 $\left(\frac{a}{p}\right) = -1$,则 $x^2 \equiv a \pmod{p}$ 无解,故 $x^2 \equiv a \pmod{p^\alpha}$ 无解。
        
    (2) 若 $\left(\frac{a}{p}\right) = 1$,则 $x^2 \equiv a \pmod{p}$ 有解,由Hensel引理知 $x^2 \equiv a \pmod{p^\alpha}$ 恰有两解且在模 $p^\alpha$ 意义下互为相反数。
    
    (3) 若 $\left(\frac{a}{p}\right) = 0$,设 $a = p^k b$,其中 $1 \leq k < \alpha$ 且 $p \nmid b$
    
    1) 若 $2 \nmid k$,则方程无解。
    
    2) 若 $2 \mid k$,设 $k = 2m$,令 $x = p^m y$,则
    \begin{align*}
    x^2 &\equiv a \pmod{p^\alpha}\\
    &\Longleftrightarrow p^{2m} y^2 \equiv p^{2m} b \pmod{p^\alpha}\\
    &\Longleftrightarrow y^2 \equiv b \pmod{p^{\alpha-2m}}
    \end{align*}
    同上,若 $\left(\frac{b}{p}\right) = 1$,则恰有两解,设 $y_0^2 \equiv b \pmod{p^{\alpha-2m}}$,
    
    则 $x_1 \equiv p^m y_0 \pmod{p^\alpha}$,$x_2 \equiv p^m(p^{\alpha-2m} - y_0) \pmod{p^\alpha}$。
    
    若 $\left(\frac{b}{p}\right) = -1$,则方程无解。
\end{solution}

    \item[3] 分别写出 $7,13,29,37$ 的全体二次剩余和非剩余.

\begin{solution}
    \small 
    \begin{tabular}{|c|>{\raggedright\arraybackslash}p{5.5cm}|>{\raggedright\arraybackslash}p{5.5cm}|}
        \hline
        & 二次剩余 & 二次非剩余 \\
        \hline
        7 & 1, 2, 4 & 3, 5, 6 \\
        \hline
        13 & 1, 3, 4, 9, 10, 12 & 2, 5, 6, 7, 8, 11 \\
        \hline
        29 & 1, 4, 5, 6, 7, 9, 13, 16, 20, 22, 23, 24, 25, 28 & 2, 3, 8, 10, 11, 12, 14, 15, 17, 18, 19, 21, 26, 27 \\
        \hline
        37 & 1, 3, 4, 7, 9, 10, 11, 12, 16, 21, 25, 26, 27, 28, 30, 33, 34, 36 & 2, 5, 6, 8, 13, 14, 15, 17, 18, 19, 20, 22, 23, 24, 29, 31, 32, 35 \\
        \hline
    \end{tabular}
\end{solution}

    \item[4] 设 $p>2$ 为奇素数,证明:
    \begin{enumerate}
        \item $\left(\frac{-1}{p}\right)=1$ 的充要条件是 $p=4 n+1$ ;
        \item $\left(\frac{2}{p}\right)=1$ 的充要条件是 $p=8 n \pm 1$ ;
        \item $\left(\frac{-2}{p}\right)=1$ 的充要条件是 $p=8 n+1,8 n+3$ ;并由此进一步证明对任意素数 $p,-1,-2,2$ 中必有一个是 $p$ 的平方剩余.
    \end{enumerate}

    \begin{proof}
        \begin{enumerate}
            \item $\left(\frac{-1}{p}\right) = (-1)^{\frac{p-1}{2}} = 1 \Leftrightarrow p = 4n+1$
        
            \item $\left(\frac{2}{p}\right) = (-1)^{\frac{p^2-1}{8}} = 1 \Leftrightarrow p = 8n \pm 1$
    
            \item  $\left(\frac{-2}{p}\right) = \left(\frac{-1}{p}\right)\left(\frac{2}{p}\right) = (-1)^{\frac{(p-1)(p+5)}{8}}$
        
            $\left(\frac{-2}{p}\right) = 1 \Leftrightarrow p = 8n+1, 8n+3$
        \end{enumerate} 
        因此,对于任意奇素数 $p$, $-1, -2, 2$ 中必有一个是 $p$ 的二次剩余。
    \end{proof}

    \item[7] 设 $x, y$ 为整数,$(x, y)=1$ ,问:$x^2+y^2$ 的大于 2 的素因子一定具有什么形式?$x^2+2 y^2$ 的大于 2 的素因子一定具有什么形式?

\begin{solution}
    (1)
\begin{align*}
    & x^2 \equiv -y^2 \pmod{p}\\
    &\Leftrightarrow \left(\frac{-1}{p}\right) = 1\\
    &\Leftrightarrow p = 4n+1
\end{align*}

(2)
\begin{align*}
    & x^2 \equiv -2y^2 \pmod{p}\\
    &\Leftrightarrow \left(\frac{-2}{p}\right) = 1\\
    &\Leftrightarrow p = 8n+1, 8n+3
\end{align*}
\end{solution}
\end{enumerate}