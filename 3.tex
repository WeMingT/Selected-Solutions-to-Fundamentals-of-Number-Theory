\section{第三周作业}
\subsection*{第二章}

\begin{enumerate}
    \item[8] 试证:当 $\omega(n)>1$ 时,$\sum_{d \mid n} \mu(d) \log d=0 ;$ 一般若 $m \geqslant 1$ 且 $\omega(n)>m$ ,则 $\sum_{d \mid n} \mu(d) \log ^m d=0$.
    \begin{proof}
        由于若$d$存在平方因子,则$\mu(d)=0$,不妨设
        \[n=p_1p_2\cdots p_k\]
        其中$p_i,p_j(i\neq j)$为互异素因子。
        \begin{enumerate}
            \item 当$\omega(n)>1$时,$\sum\limits_{d|n}\mu(d)\log d=0$。\\
            $k=\omega(n)>1\implies k>1$
            \begin{align*}
                \sum\limits_{d|n}\mu(d)\log d&=\sum\limits_{S\subseteq\{p_1,p_2,\cdots,p_k\}}\mu(\prod\limits_{p\in S}p)\log(\prod\limits_{p\in S}p)\\
                &=\sum\limits_{S\subseteq\{p_1,p_2,\cdots,p_k\}}(-1)^{|S|}\sum\limits_{p\in S}\log p
            \end{align*}
            交换求和次序,有:
            \[\sum\limits_{d|n}\mu(d)\log d=\sum\limits_{p|n}\log p\sum\limits_{\substack{d \mid n \\ p|d}}\mu(d)\]
            固定$p$,则
            \begin{align*}
                \sum\limits_{\substack{d \mid n\\ p \mid d}}\mu(d)&=(-1)\left[(-1)^0\binom{k-1}{0}+(-1)^1\binom{k-1}{1}+\cdots+(-1)^{k-1}\binom{k-1}{k-1}\right]\\
                &=(-1)(-1+1)^{k-1}\\
                &=0
            \end{align*}
            因此,
            \[\sum\limits_{d|n}\mu(d)\log d=0\]
            \item 若$m \geq 1$,且$\omega(n)>m$,则$\sum\limits_{d|n}\mu(d)\log^m d=0$。\\
            $k=\omega(n)>m\geq 1\implies k>1$
            \[\sum\limits_{d|n}\mu(d)\log^m d=\sum\limits_{d|n}\mu(d)\sum\limits_{\substack{
                p_1|d,\cdots,p_m|d \\ p_{i} \mid n}}\log p_1\cdots\log p_m\]
            交换求和次序,有:
            \[\sum\limits_{d|n}\mu(d)\log^m d=\sum\limits_{p_1|n}\cdots\sum\limits_{p_m|n}\log p_1\cdots\log p_m\sum\limits_{\substack{d \mid n\\ p_1|d,\cdots,p_m|d} }\mu(d)\]
            固定$p_1,p_2,\cdots,p_m$,令$S=\{s:s=p_i,i=1,2,\cdots,m\}$为其中不同素因子的集合;设$r=|S|$,则$r\leq m<k.$\\
            则:
            \begin{align*}
                &\sum\limits_{\substack{d \mid n \\p_1|d,\cdots,p_m|d}}\mu(d)\\
                &=(-1)^r\left[(-1)^0\binom{k-r}{0}+(-1)^1\binom{k-r}{1}+\cdots+(-1)^{k-r}\binom{k-r}{k-r}\right]\\
                &=(-1)^r(-1+1)^{k-r}\\
                &=0
            \end{align*}
            因此,
            \[\sum\limits_{d|n}\mu(d)\log^m d=0\]
        \end{enumerate}
    \end{proof} 
    \item[10] 求 $\sum_{n=1}^{\infty} \mu(n!)$ 之值.
    \begin{solution}
      对于$n\geq 4$,有$2^2=4|n$,故$\mu(n)=0$。\\
则:
\begin{equation*}
    \sum\limits_{n=1}^\infty\mu(n!)=\mu(1)+\mu(2)+\mu(6)=1+(-1)+(-1)^2=1
\end{equation*}
    \end{solution}
    \item[11] 证明:$\sum_{d \mid n} \mu^2(d)=2^{\omega(n)}$ 及 $\sum_{t \mid n} \mu(t) d(t)=(-1)^{\omega(n)}$ .
    \begin{proof}
        \begin{enumerate}
            \item 设$f=\mu^2 * u$,则$f$为积性函数,且$f(n)=\sum\limits_{d|n}\mu^2(d)$。\\
            设$n=p^\alpha$,则有:
            \[f(p^\alpha)=\begin{cases}
                1, & \alpha=0\\
                2, & \alpha\geq 1
            \end{cases}\]
            若$m=p_1^{\alpha_1}p_2^{\alpha_2}\cdots p_s^{\alpha_s}$,则
            \[f(m)=f(p_1^{\alpha_1})f(p_2^{\alpha_2})\cdots f(p_s^{\alpha_s})=2^{\omega(m)}\]
            即
            \[\sum\limits_{d|n}\mu^2(d)=2^{\omega(n)}\]
            
            \item 设$g=\mu d * u$,则$g$为积性函数,且$g(n)=\sum\limits_{t|n}\mu(t)d(t)$。\\
            设$n=p^\alpha$,则有:
            \[g(p^\alpha)=\begin{cases}
                1, & \alpha=0\\
                -1, & \alpha\geq 1
            \end{cases}\]
            若$m=p_1^{\alpha_1}p_2^{\alpha_2}\cdots p_s^{\alpha_s}$,则
            \[g(m)=g(p_1^{\alpha_1})g(p_2^{\alpha_2})\cdots g(p_s^{\alpha_s})=(-1)^{\omega(m)}\]
            即
            \[\sum\limits_{t|n}\mu(t)d(t)=(-1)^{\omega(n)}\]
        \end{enumerate}
    \end{proof}
    \item[12] 试证:$\sum_{d \mid n} \mu(d) \sigma(d)=(-1)^{\omega(n)} \prod_{p \mid n} p$ 及 $\sum_{d \mid n} \mu(d) \varphi(d)=(-1)^{\omega(n)} \prod_{p \mid n}(p-2)$ .
    \begin{proof}
        \begin{enumerate}
            \item $\sum\limits_{d|n}\mu(d)\sigma(d)=(-1)^{\omega(n)}\prod\limits_{p|n}p$。\\
            设$f=\mu*\sigma$,则$f$为积性函数,且$f(n)=\sum\limits_{d|n}\mu(d)\sigma(d)$。\\
            设$n=p^\alpha$,则有:
            \[f(p^\alpha)=\begin{cases}
                1, & \alpha=0\\
                -p, & \alpha\geq 1
            \end{cases}\]
            若$m=p_1^{\alpha_1}p_2^{\alpha_2}\cdots p_s^{\alpha_s}$,则
            \[f(m)=f(p_1^{\alpha_1})f(p_2^{\alpha_2})\cdots f(p_s^{\alpha_s})=(-1)^{\omega(m)}\prod\limits_{p|m}p\]
            即
            \[\sum\limits_{d|n}\mu(d)\sigma(d)=(-1)^{\omega(n)}\prod\limits_{p|n}p\]
            
            \item $\sum\limits_{d|n}\mu(d)\varphi(d)=(-1)^{\omega(n)}\prod\limits_{p|n}(p-2)$。\\
            设$g=\mu*\varphi$,则$g$为积性函数,且$g(n)=\sum\limits_{d|n}\mu(d)\varphi(d)$。\\
            设$n=p^\alpha$,则有:
            \[g(p^\alpha)=\begin{cases}
                1, & \alpha=0\\
                2-p, & \alpha\geq 1
            \end{cases}\]
            若$m=p_1^{\alpha_1}p_2^{\alpha_2}\cdots p_s^{\alpha_s}$,则
            \[g(m)=g(p_1^{\alpha_1})g(p_2^{\alpha_2})\cdots g(p_s^{\alpha_s})=(-1)^{\omega(m)}\prod\limits_{p|m}(p-2)\]
            即
            \[\sum\limits_{d|n}\mu(d)\varphi(d)=(-1)^{\omega(n)}\prod\limits_{p|n}(p-2)\]
        \end{enumerate}
    \end{proof}
    \item[14] (1)设 $n>1$ ,证明:$\sum_{\substack{1 \leqslant d \leqslant n \\(n, d)=1}} d=\frac{1}{2} n \varphi(n)$ ;\\
    (2)设 $n$ 为奇数,证明:$\sum_{\substack{1 \leqslant d \leqslant \frac{n}{n} \\(d, n)=1}} d=\frac{1}{8} n \varphi(n)-\frac{1}{8} \prod_{p \mid n}(1-p)$ .
    \begin{proof}
        \begin{enumerate}
            \item 若$n>1$,则
            \[\sum\limits_{\substack{1\leq d\leq n\\(n,d)=1}}d=\frac{1}{2}\sum\limits_{\substack{1\leq d\leq n\\(n,d)=1}}d+\frac{1}{2}\sum\limits_{\substack{1\leq d\leq n\\(n,d)=1}}(n-d)\]
            \[=\frac{1}{2}\sum\limits_{\substack{1\leq d\leq n\\(n,d)=1}}n\]
            \[=\frac{1}{2}n\varphi(n)\]
            
            \item 设$S=\sum\limits_{\substack{1\leq d\leq \frac{n}{2}\\(d,n)=1}}d$,则有
            \[S=\sum\limits_{d=1}^{\lfloor \frac{n}{2} \rfloor}d\cdot I((d,n))\]
            其中$I=\mu*u$,故
            \[I((d,n))=\sum\limits_{k|(d,n)}\mu(k)\]
            于是,
            \[S=\sum\limits_{d=1}^{\lfloor \frac{n}{2} \rfloor}d\sum\limits_{k|(d,n)}\mu(k)\]
            交换求和次序,有
            \[S=\sum\limits_{k|n}\mu(k)\sum\limits_{\substack{1\leq d\leq \lfloor \frac{n}{2} \rfloor\\k|d}}d\]
            固定$k$,设$d=km$,则
            \[\sum\limits_{\substack{1\leq d\leq \lfloor \frac{n}{2} \rfloor\\k|d}}d=k\sum\limits_{m=1}^{\lfloor\frac{n}{2k}\rfloor}m=k\cdot\frac{M(M+1)}{2}\]
            其中,$M=\lfloor\frac{n}{2k}\rfloor$。
            
            而
            \[\lfloor\frac{n}{2k}\rfloor=\lfloor\frac{\frac{n}{k}}{2}\rfloor=\frac{\frac{n}{k}-1}{2}\]
            因此,
            \[\sum\limits_{\substack{1\leq d\leq \lfloor \frac{n}{2} \rfloor\\k|d}}d=k\cdot\frac{\frac{\frac{n}{k}-1}{2}(\frac{\frac{n}{k}-1}{2}+1)}{2}\]
            \[=\frac{n^2-k^2}{8k}\]
            则
            \[S=\sum\limits_{k|n}\mu(k)\cdot\frac{n^2-k^2}{8k}\]
            \[=\frac{1}{8}\left(n^2\sum\limits_{k|n}\frac{\mu(k)}{k}-\sum\limits_{k|n}\mu(k)k\right)\]
            其中,
            \[\sum\limits_{k|n}\frac{\mu(k)}{k}=\frac{\varphi(n)}{n}\]
            \[\sum\limits_{k|n}\mu(k)k=\prod\limits_{p|n}(1-p)\]
            于是,
            \[S=\frac{1}{8}\left(n^2\cdot\frac{\varphi(n)}{n}-\prod\limits_{p|n}(1-p)\right)\]
            \[=\frac{1}{8}n\varphi(n)-\frac{1}{8}\prod\limits_{p|n}(1-p)\]
        \end{enumerate}
    \end{proof}
    \item[16] 求出所有使 $\varphi(n)=24$ 的自然数.
    \begin{solution}
      对于$n\in\mathbb{N}^+$,作如下素因数分解
      \[n=p_1^{\alpha_1}p_2^{\alpha_2}\cdots p_k^{\alpha_k}\]
      则
      \[\varphi(n)=n\prod\limits_{p|n}(1-\frac{1}{p})\]
      \[=p_1^{\alpha_1-1}(p_1-1)p_2^{\alpha_2-1}(p_2-1)\cdots p_k^{\alpha_k-1}(p_k-1)\]
      注意到:$24=2^3\times 3$
      \begin{enumerate}
        \item $n=p^k$\\
        即
        \[p^{k-1}(p-1)=2^3\times 3\]
        无解。
        \item $n=p^aq^b$\\
        即
        \[p^{a-1}(p-1)q^{b-1}(q-1)=2^3\times 3\]
        \begin{enumerate}
          \item $a=1$,$b=1$,则
          \[(p-1)(q-1)=24\]
          而$24=1\times 24=2\times 12=3\times 8=4\times 6$\\
          又$p,q$均为素数,故
          \[(p,q)=(3,13),(5,7)\]
          于是
          \[n=pq=39,35\]
          \item $a=2$,$b=1$,则
          \[p(p-1)(q-1)=24\]
          故
          \[(p,q)=(2,13),(3,5)\]
          于是
          \[n=p^2q=52,45\]
          \item $a=3$,$b=1$,则
          \[p^2(p-1)(q-1)=24\]
          故
          \[(p,q)=(2,7)\]
          于是
          \[n=p^3q=56\]
          \item $a=3$,$b=2$,则
          \[p^2(p-1)q(q-1)=24\]
          故
          \[(p,q)=(2,3)\]
          于是
          \[n=p^3q^2=72\]
          \item 其他情况,均无解。
        \end{enumerate}
        \item $n=p^aq^br^c$\\
        即
        \[p^{a-1}(p-1)q^{b-1}(q-1)r^{c-1}(r-1)=24\]
        \begin{enumerate}
          \item $a=b=c=1$,则
          \[(p-1)(q-1)(r-1)=24\]
          而$24=1\times 2\times 12=1\times 3\times 8=1\times 4\times 6=2\times 3\times 4$\\
          故
          \[(p,q,r)=(2,3,13),(2,5,7)\]
          于是
          \[n=pqr=78,70\]
          \item $a=2$,$b=c=1$,则
          \[p(p-1)(q-1)(r-1)=24\]
          故
          \[(p,q,r)=(2,3,7),(3,2,5)\]
          于是
          \[n=p^2qr=84,90\]
          \item $a=b=2$,$c=1$或其它情况,均无解。
        \end{enumerate}
        \item $n=p^aq^br^cs^t.$

        因为若$n=2\times 3\times 5\times 7=210$,则$\varphi(n)=48>24$,故无解。
    \end{enumerate}
        综上所述,$n$所有可能的取值为
        \[39,35,52,45,56,72,78,70,84,90\]
        共10种。
    \end{solution}
    \item[19] 求出所有 $4 \nmid \varphi(n)$ 的自然数 $n$ .
    \begin{solution}
      对于$n\in\mathbb{N}^+$,作如下素因数分解
      \[n=p_1^{k_1}p_2^{k_2}\cdots p_m^{k_m}\]
      则
      \[\varphi(n)=p_1^{k_1-1}(p_1-1)p_2^{k_2-1}(p_2-1)\cdots p_m^{k_m-1}(p_m-1)\]
      设$n=2^a\cdot m$,其中$2^a \mid n$而$2^{a+1} \nmid m$。

      由于$\varphi(n)$为积性函数,于是
      \[\varphi(n)=\varphi(2^a)\cdot\varphi(m)\]
      \begin{enumerate}
        \item $a=0$,则$n$为奇数。\\
        对于$m$作素因数分解
        \[m=p_1^{b_1}p_2^{b_2}\cdots p_t^{b_t}\]
        其中$p_i$为奇质数。
        \begin{enumerate}
          \item 若$p_i\equiv 1\pmod{4}$,则$p_i-1\equiv 0\pmod{4}$,则
          \[4|\varphi(m)\]
          \item 若$p_i\equiv 3\pmod{4}$,则
          \[\varphi(p_i^{b_i})=p_i^{b_i-1}(p_i-1)\equiv 2\pmod{4}\]
          若$t \leq 2,$则存在$i,j$使得$4|(p_i-1)(p_j-1)$,故$4|\varphi(n)$。\\
          若$t=0,$则$n=1$,$\varphi(1)=1$,有$4\nmid\varphi(1)$。\\
          若$t=1$,则$n=p^b$,其中$p\equiv 3\pmod{4}$,故$\varphi(p^b)\equiv 2\pmod{4}$。
        \end{enumerate}
        \item $a=1$,则
        \[\varphi(n)=\varphi(2\cdot m)=\varphi(2)\varphi(m)=\varphi(m)\]
        与上一种情况类似,故
        \[4\nmid\varphi(n)\Longleftrightarrow n=2 \text{或}n=2\cdot p^k\]
        其中$p\equiv 3\pmod{4}$,$k\geq 1$。
        \item $a=2$,则
        \[\varphi(n)=\varphi(4\cdot m)=\varphi(4)\varphi(m)=2\varphi(m)\]
        比较可知,$4\nmid\varphi(m)\Longleftrightarrow m=1\Longleftrightarrow n=4$。
        \item $a\geq 3$,则$4|\varphi(n)$,无解。
      \end{enumerate}
      综上所述,$n$所有可能的取值为
      \[1,2,4,p^k,2\cdot p^k\]
      其中$p$为素数且$p\equiv 3\pmod{4}$,$k\geq 1$。
    \end{solution}
    \item[22] 设 $\Lambda(n)$ 为 Mangoldt 函数,且 $\psi(x) = \sum_{n\leq x} \Lambda(n)$,则
\[\sum_{n\leq x}\psi\left(\frac{x}{n}\right) = \sum_{n\leq x}\Lambda(n)\left[\frac{x}{n}\right] = \sum_{n\leq x}\log n.\]
    \begin{proof}
        \begin{enumerate}
            \item 
            \begin{align*}
                \sum\limits_{n\leq x}\psi\left(\frac{x}{n}\right) &= \sum\limits_{n\leq x}\sum\limits_{m\leq \frac{x}{n}}\Lambda(m)\\
                &= \sum\limits_{m\leq x}\Lambda(m)\sum\limits_{n\leq \frac{x}{m}}1\\
                &= \sum\limits_{m\leq x}\Lambda(m)\left[\frac{x}{m}\right]
            \end{align*}
            
            \item 
            \begin{align*}
                \sum\limits_{n\leq x}\log n &= \sum\limits_{n\leq x}\sum\limits_{d|n}\Lambda(d)\\
                &= \sum\limits_{d\leq x}\Lambda(d)\sum\limits_{k\leq \frac{x}{d}}1\\
                &= \sum\limits_{d\leq x}\Lambda(d)\left[\frac{x}{d}\right]
            \end{align*}
        \end{enumerate}
    \end{proof}
    \item[24]  设 $\sigma(n)$ 为除数和函数,证明:\\
    (1)$\sigma(n)=n+1$ 的充要条件是 $n$ 为素数;\\
    (2)如果 $n$ 为完全数,即 $\sigma(n)=2 n$ ,则
\begin{equation*}
    \sum_{d \mid n} \frac{1}{d}=2
\end{equation*}
    \begin{proof}
        \begin{enumerate}
            \item 必要性显然。\\
            充分性:\\
            若$n=1$,则$\sigma(1)=1$,而$1+n=2$,故不满足。\\
            若$n$为合数,则存在$d$($d\neq 1$且$d\neq n$)s.t. $d|n$。\\
            而
            \[\sigma(n)\geq 1+d+n>1+n\]
            故$\sigma(n)\neq 1+n$,矛盾。\\
            因此,$n$为素数。
            
            \item 
            \[\sum\limits_{d|n}\frac{1}{d}=\frac{1}{n}\sum\limits_{d|n}\frac{n}{d}=\frac{1}{n}\sum\limits_{d|n}d=\frac{\sigma(n)}{n}=\frac{2n}{n}=2\]
        \end{enumerate}
    \end{proof}
\end{enumerate}